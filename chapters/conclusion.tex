\chapter{Conclusion}
\label{ch:conclusion}
During accomplishing the thesis work I aimed to supply an explicit overview of image medical background including types of medical image modalities together with how usually they are reconstructed after acquisition, narrated about classical and modern segmentation approaches in addition describing mathematical background behind them. Other than that I pointed out pros and cons of each reviewed algorithm, also evaluated them on the baseline CT scan sample to demonstrate the visualisation of their performances. 

Following that, I have spotted detailed overview of vertebral segmentation and labelling problem highlighting the historical research way from 2015 to nowadays.  

Afterwards I switched the direction on the proposed solution itself. Altogether with dataset description, metrics and losses mathematical overview I conveyed the models architectures. Apprised how they were trained other than that how the data was augmented. To validate the obtained results I introduced the competitor and within comparison against participant I demonstrated the increase of the performance of my approach.

Precis, 2-stage U-net pipeline which initially detect vertebrae then identify specific vertebrae keening with a 2-class cross entropy and $l^{1}$ loss accordingly, outcomes spectacular results for vertebrae segmentation problem on a publicly available pathological spine CT data set. 