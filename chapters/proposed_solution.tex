\chapter{Proposed solution}
\label{ch:proposed_solution}
In this section I will deliberate the proposed model background and its main dissemblance from general unet architecture as well as narrate the details of the proposed approach. 

Jointly the solution it was utilized 2 convolution networks with same architectures but sundry purposes and parameters. By scenario firstly it is issued detection model afterwards identification model. 

The detection model comes up with segmenting of vertebra from background. Whereas the identification model identifies what vertebrates belong which segmented pixels. To fine it down the identification model does not categorize each pixel in discrete manner, vice continuously produce value which subsequently is rounded correspondingly to a specific vertebra. It worth to mention inwardly full solution captures both short-range and long-range information. Short-range information is retrieved utilizing detection model. The identification model is trained by feeding in large slices which capture long-range information which is essential for the task of identifying individual vertebrae. 

Acquisition of final results is done by multiplying detection and identification models results to elaborate labels on each pixel afterwards aggregating them to produce final centroid estimates for each vertebra.

The visual step by step representation of results is demonstrated on the figure \ref{fig:detection_identification_steps}. The chain of steps will not be capable if there would not be included background pixels in the identification stage therefore segmenting out the background in the detection step.

\begin{figure}[h]
    \centering \includegraphics[width=12cm]{images/detection_identification_steps.png}
    \caption {(a) shows an original scan in grey-scale, (b) shows the output of the detection model applied to the scan, (c) shows the output of the identification model applied to the scan, (d) shows (b) and (c) multiplied together to produce a final prediction for each pixel.}
    \label{fig:detection_identification_steps}
\end{figure}

\section{Terminology} 
In this section I will introduce terms frequently used in sections below: "FCN" - fully connected neural network, "CNN" - convolutional neural network, "detection model" - neural network which stands for set of vertebra apart background, "identification model" - identifies which vertebrate belongs which segmented pixel, "Dice score" - Dice similarity coefficient to quantify the goodness of identification model predictions, "SSIM" - structural similarity index measure, "NRMSE" - normalized root mean square error measure, "ID rate" - measurement of correctness of detection model predictions.

\section{Dataset}
For dataset it was used \href{https://biomedia.doc.ic.ac.uk/}{\color{blue}"BioMediaA"} database provide spine-focused CT scans of 125 patients with different types of pathologies. In total dataset consists of 242 scans. Each scan was manually annotated in terms of vertebrae centroids. As well, based on description of dataset by \href{https://biomedia.doc.ic.ac.uk/}{\color{blue}"BioMediaA"}, the mapping of a class-label-to-vertebra-label is fixed, inherently implying labelling the vertebra.    

Eventually data set was splitted into dedicated training (with ground-truth labels) and test sets each 242 and 60 scans accordingly. The scans were re-sampled at $1mm \cdot 1mm \cdot 1mm$ resolution meaning each pixel in the generated samples from the scans provided a 3 dimensional $1mm$ section of the patient spine.

In the case of the detection model input images contain two values: 0 representing background and 1 representing vertebrae. The identification model’s input contains values from 0 to 26. Whilst 0 representing background, 1 representing C1 vertebrae up to 26 representing S2 vertebrae. 

\begin{figure}[h]
    \centering \includegraphics[width=10cm]{images/labeled_data.png}
    \caption {(a) shows a dense labelling, (b) an example of a sample used to train the detection model, (c) an example of a sample used to train the identification model.}
    \label{fig:labeled_data}
\end{figure}

For identification model the size of the sample is 8 x 80 x 320, if the original scan is not larger enough to fill those dimensions some padding is added, as can be seen at the top.

\section{Metrics}
As for metrics I utilized multiple ones such as Dice score, Id rate, mean localization distance and standard deviation (std) distance. On top of that I established Jaccard index, SSIM and NRMSE. Dice score measures goodness of predictions matched the training annotated ground truth segmentation, Id rate is the percentage of the centroid estimations predicted which are closest to the correct ground truth vertebra centroid (and are less than 20mm from that centroid). Mean and Std relates to the localization error distance between the predicted centroid positions and the ground-truth centroid positions for the same vertebrae (if it occurs in the scan). Jaccard index stands for the area of overlap between the predicted segmentation and the ground truth image divided by the area of union between the predicted segmentation and the ground truth image. SSIM index is for measuring similarity based on calculating various windows of an image. And NRMSE is normalized root mean square error metric. 

\subsection{Dice score}
Recall Vikas Thada and Vivek Jaglan \cite{Thada2013}, Dice similarity coefficient (Dice score) was used to quantify how closely identification predictions matched the training annotated ground truth segmentation. 

Assume we annotated some ground truth region in particular image and then make an automated algorithm to do it so far. To validate the algorithm it is calculated Dice score, which is a measure of how similar the objects are. In other words it is the size of the overlap of the two segmentations divided by the total size of the two objects. As for mathematical representation it is formed as:
\begin{align*}
  \text{Dice score} = \frac{2\cdot | A\cap B|}{|A| + |B|}
\end{align*}

\subsection{Jaccard Index}
The Intersection-Over-Union (IoU), also known as the Jaccard Index, is one of the most commonly used metrics in image segmentation problems. Jaccard index is the area of overlap between the predicted segmentation and the ground truth image divided by the area of union between the predicted segmentation and the ground truth image. Mathematically the notion if defined as: 
\[ J(A,B) = \frac{|A \cap B|}{|A \cup B|} = \frac{|A \cap B|}{|A| + |B| - |A \cup B|}\]

\subsection{ID rate}
It was employed ID rate measurement to validate correctness of closeness of predicted centroid estimations with ground truth vertebra centroid. The correctness measure actually is considered as a distance and was calculated for predicted and ground truth vertebrae centroids accordingly as:
\[ dist = ||A||_F = [\sum_{i,j} \abs(a_{i,j})^2]^{\frac{1}{2}} \]

Afterwards it was compared with predefined threshold as:
\begin{align*}
\begin{cases}
\text{dist, } \mbox{ if } \text{dist} \leq \text{threshold (min dist = 20 mm)} \\ 
\text{skip}, \mbox{ otherwise} \end{cases}
\end{align*}

\subsection{Localization error}
As for localisation error, if centroid occurred at the scan it was deliberated by mean and std of distance of localization error between predicted centroid position and ground-truth centroid position for particular vertebrae. Localization error notion is such as:
\[ \sum_{k=1}^K \frac{||\hat{\zeta_k} - \zeta_k || }{K}, \]
where $K$ is the source number, $\hat{\zeta_k}$ and $\zeta_k$ are the estimated location and the real location of the $k$-th source, respectively.

\subsection{SSIM}
The structural similarity index measure (SSIM) was proposed to measure the similarity between two images (ground truth and predicted). The SSIM index is calculated on various windows of an image. The measure between two windows  $x$ and $y$ of common size $N \times N$ is calculated as: 
\[ \text{SSIM}(x,y) = \frac{(2\mu_x\mu_y + c_1)(2\sigma_{xy}+c_2)}{(\mu_x^2+\mu_y^2+c_1)(\sigma_x^2+\sigma_y^2+c_2)}\]

The SSIM values ranges between $0$ to $1$. Whereas $1$ means perfect match the reconstruct image with original one.

\subsection{NRMSE}
Normalized root mean square error relates to the RMSE, meaning it can be interpreted as a fraction of the overall range that is typically resolved by the model. In mathematical terms it is defined as follows:
\[ \text{NRMSE} = \sqrt{\frac{1}{n} \sum_{i=1}^n\left(\log \frac{\hat{y}+1}{y+1}\right)^2} \]


\section{Loss function}
Reminding, loss function is a function that maps an event or values of one or more variables onto a real number intuitively representing some "cost" associated with the event. 

Typically, with neural networks, we seek to minimize the error. As such, the objective function is often referred to as a cost function or a loss function and the value calculated by the loss function is referred to as simply loss.

\subsection{Weighted Categorical Cross Entropy}
Categorical Cross Entropy stands for quantifying multi-class classification tasks. As it all, it evaluates the difference between two probability distributions. Mathematically, categorical cross entropy has following notion: \[L(P, Q) = - \sum_{x \in X} P(x) \log Q(x), \] 
where $P$ and $Q$ are certain probabilities. 

Within the detection model it was used modified version of categorical cross entropy such as weighted categorical cross entropy. Mathematically loss function is defined as:
\begin{align*}
 L(P, Q) = P(0)\log Q(0) + P(1)\log Q(1)
\end{align*}

To reflect the proportion of background labels in the samples, the background label and vertebrae labels were issued with additional weights coefficients as $0.1$ and $0.9$ accordingly the modified weighted categorical cross entropy was defined as:
\begin{align*}
 L(P, Q) = 0.1 \cdot P(0)\log Q(0) + 0.9 \cdot P(1)\log Q(1)
\end{align*}

\subsection{$l^{1}$}
$l^{1}$ loss function stands for least absolute deviations. It is used to minimize the error which is the sum of the all the absolute differences between the true value and the predicted value. Mathematically it is defined as: 
\[ \text{$l$1} = \sum_{i=1}^n |y - \hat{y}| , \]
where $\hat{y}$ - predicted value and $y$ - true value.

Loss was utilized within the identification model which does not classify each pixel discretely but instead produces a continuous value for each pixel. This value is then rounded to an integer which corresponds to a specific vertebra, for example $4 = C4$ vertebra. 

I used $l^{1}$ on each pixel and thus captured the ordering through this loss function. For that I have used the modified version of it set as: 
\begin{align*}
 \text{$l^{1}$} = \begin{cases} \lvert \hat{y_i} - y_i \rvert \mbox{, if } x\mbox{$\neq 0$} \\ 0, \mbox{ otherwise} \end{cases}
\end{align*}


\section{Detection Model}
As previously mentioned, detection model comes up with segmenting of vertebra from background. The detection model architecture is shown on Figure \ref{fig:detection_model}.

\begin{figure}[h]
    \centering \includegraphics[width=10cm]{images/detection_model.png}
    \caption {Detection Model Architecture}
    \label{fig:detection_model}
\end{figure}

Taking it closer into consideration the model classifies each dedicated pixel of scan whether it is background or vertebrae. Jointly the initialization of weights it was  defined to point out background labels with $0.1$ weighting and the vertebrae label with $0.9$ weighting. As well, it was thought-out to repeal the fraction of nature background state at particular scan. As for loss function it was consumed weighted categorical cross entropy with 2 classes, defined as:
\begin{align*}
 L(P, Q) = 0.1 \cdot P(0)\log Q(0) + 0.9 \cdot P(1)\log Q(1),
\end{align*}
where P, Q - particular distributions 

Unlike U-net architecture it was used same padding for all convolutional layers. Other properties was as: stride $1$, a learning rate $\lambda$ of $0.001$, a batch size of $16$ and batch normalization after every convolutional layer with momentum of $0.1$. As for optimizer it was approached "Adam".
 
\subsection{Detection augmentation}
Unto passing particular batch of scans to detection model each image was represented as 5 cropped samples intruding that at least 4 of them contain some vertebrae pixels. In the meantime each sample scan was sized as $64 \cdot 64 \cdot 80$, meaning 64 pixels by 64 pixels by 80 channels. Withal accordingly each sample has an accompanying dense labelling, containing 0s (for background) and 1s (for vertebrae) of the same size.    
 
\section{Identification Model}
Identification model stands for identifies what vertebrates belong which segmented pixels. The identification model architecture is shown on Figure \ref{fig:identification_model}. 

\begin{figure}[h]
    \centering \includegraphics[width=11cm]{images/identification_model.png}
    \caption {Identification Model Architecture}
    \label{fig:identification_model}
\end{figure}

To apply identification it was established 2D U-net  architecture with revamped parameters. Updates brought in changes on number of channels of input layer which corresponds to $8$ to pass in samples of size $8 \cdot 80 \cdot 320$ from detection model step. 

Introducing filters sized $5 \cdot 20$ at the lowest level of the architecture  increases the size of the receptive field thereby maximizing the potential information retrieved by the network. To clarify, the background pixels are strained out, denoting just that pixels which are labeled as background when we train model. 

As for loss function I used $l^{1}$. I used same padding for all convolutional layers with stride $1$, a learning rate
$\lambda$ of $0.001$, a batch size of $32$ and batch normalization after every convolutional layer with momentum of $0.1$. As optimizer I used "Adam" samely as for detection model.  

The identification model outputs a continuous value for each pixel of a scan which is rounded using $l^{1}$ loss to an integer to correspond to a specific vertebrae. The identification model gives a value to each pixel even if that pixel is not depicting any vertebrae and is a background. These pixels will be filtered out of the final predictions by multiplying them by their corresponding pixel from the output of the detection model which should be 0 for a background pixel. 

\subsection{Identification augmentation}
Altogether proceeding with identification network, for each scan in the training set it was generated 100 cropped samples sized $8 \cdot 80 \cdot 320$ coercing all of them retrieving some vertebrae pixels. 

\section{Training models}
To recap, input to detection model was initially $64 \cdot 64 \cdot 80$ particular scan which was transformed (5 crops) what afterwards got shape of $32 \cdot 32 \cdot 40$. It worth to mention, jointly training process it was applied padding which discarded the outer border of scan producing $16 \cdot 16 \cdot 20$ shape so far. In simplified terms, it reduced edge artifacts in the detection and led to improved mean localization scores. $27$ hours the detection model had been training for $50$ epochs. The validation Dice score achieved $0.9613$, SSIM $0.97$, Jaccard $0.91$, NRMSE $0.35$.   

In regard of identification model, particular input was sized of $8 \cdot 80 \cdot 320$. $15$ hours the identification model had been training for $35$ epochs. At test time it was passed whole slices of single input scan, padded to the nearest and multiple of $16$.

\section{Inference}
As soon as the identification model outcomes predicted label per each pixel foremost thresholding $x_\upsilon$ is applied as validation whether include particular centroid in the prediction or not by comparing the obtained  number of pixels for a vertebra and assumed thresholding number. The threshold $x_\upsilon$ is specific to the vertebra $\upsilon$ and formed as:
\begin{align*}
  x_\upsilon = \max(3000, 0.4 \cdot R_\upsilon^3),
\end{align*}
where $R$ - the radius of the vertebra $v$.

\section{Results}
The resulting test measurements have been calculated within testing set of 60 scans of various patient.
\begin{figure}[h]
    \centering 
    \includegraphics[width=11cm]{images/detection_identification_steps.png}
    \caption{Per step visualisation of detection and identification  models predictions.}
    \label{fig:step_step_predictions}
\end{figure}
Moving from left side to right side on the Figure \ref{fig:step_step_predictions} it is represented a visualisation of predictions for a single scan within the entire pipeline. 

Looking deeper on the images of 
Figure \ref{fig:step_step_predictions}, we see that (a) shows an original scan in grey scale, (b) shows the output of the detection model applied to the scan, (c) shows the output of the identification model applied to the scan, (d) shows (b) and (c) multiplied together to produce a final prediction for each pixel.

On the Figure \ref{fig:predictions} depicted 3 resulting different predictions which represents predictions themselves as well as additional impacting information as ground-truth and localization error. 
\begin{figure}[h]
    \centering \includegraphics[width=9cm]{images/predictions.png}
    \caption {Random predictions, whereas the red points denote pipeline centroid predictions, the white points denote ground-truth centroid locations.}
    \label{fig:predictions}
\end{figure}

\newpage
\section{Competitors}
As the baseline comparison it was considered "Joint Vertebrae Identification and Localization in Spinal CT Images by Combining Short-Range and Long-Range Contextual Information" work \cite{Liao2018} done by Liao at 2018. The pipeline architecture is shown on Figure \ref{fig:competitor}.

\begin{figure}[h]
    \centering \includegraphics[width=12cm]{images/competitor.png}
    \caption {Overall architecture of the competitor method.}
    \label{fig:competitor}
\end{figure}

The overall architecture of the proposed method may be traversed as three-stage approach. In the first stage, it was applied deep multi-task 3D CNN using randomly cropped 3D vertebrae samples. Afterwards within second stage transform trained multi-task 3D CNN into a multi-task 3D fully convolutional network by converting the fully connected layers to 3D convolutional layers. Decisively within third stage the extracted sample features will be ordered and form a set of feature sequences. 

In terms of the loss function authors established following of losses defined as:
\[L = L_{id} + \lambda L_{loc}, \]
where $L_{id}$ denotes the identification loss, $L_{loc}$ denotes the localization loss and $\lambda$ denotes the importance coefficient that controls the relative learning rate of the two tasks. 

Unlike used weighted category cross entropy by me, competitor used cross entropy softmax loss, which is commonly used for classification problems, for the identification task. The mathematical notion is following:
\[L_{id}  = - \frac{1}{N} \sum_{i=0}^{N-1}\sum_{j=0}^{P-1} y_{ij} \log (f_{id}^j(x_i; W)) + (1-y_{ij}) \log (1-f_{id}^j(x_i;W)) \]

For localization loss function they used smooth $l^{1}$ loss defined as:
\[ L_{loc} = \frac{1}{m} \sum_{i=0}^{N-1} [y_i_{P-1} = 0] \sum_{j=0}^{D-1} \text{smooth}_L_1 (p_{ij} - f_{loc}^j(x_{ij}; W)), \]
where:
\begin{align*}
\text{smooth}_{L_1}(x) =
\begin{cases}
0.5x^2 \mbox{, if } \lvert x \rvert < 1  \\ 
\lvert x \rvert - 0.5 \mbox{, otherwise} \end{cases}
\end{align*}

Based on the proposed by Haofu Liao, Addisu Mesfin, and Jiebo Luo table of comparison \cite{Liao2018} of the methods I summarized following \ref{tab:2-stage_u-net_vs_rest}:

\begin{table}[h]
\centering
\begin{tabular}{|l|c|c|c|c|c|}
\hline
Metrics & Glocker & Chen & Dong & Liao & 2-stage U-net \\
\hline
\hline
ID rate & 74.0\% & 84.2\% & 85\% & 88.3\% & 91.7\% \\
\hline
Mean & 13.20 & 8.82 & 8.6 & 6.47 & 5.60 \\
\hline
Std & 17.83 & 13.04 & 7.8 & 8.56 & 7.10 \\
\hline
SSIM & - & - & - & - & 0.95 \\ 
\hline
DICE & - & - & - & - & 0.96\\ 
\hline
JACCARD & - & - & - & - & 0.92\\ 
\hline
NRMSE & - & - & - & - & 0.38 \\
\hline
\end{tabular}
\caption{Table of comparison 2-stage U-net and competitors.}
\label{tab:2-stage_u-net_vs_rest}
\end{table}

The deduction of models comparison is that "ID rate" was beated on $3.4$\% as well as localization error was decreased in terms of both mean and std. The proposed in this work method does not utilize RNN unlike competitors. The way of measuring the losses were modified, thus making the method likely more efficient and intelligible. 