\chapter{Introduction}
\label{ch:introduction}


The main part of the human axial skeleton consists of 33-34 vertebrae which coherently connected to each other in an upright position \cite{Ahlberg2005}. Vertebrates types can be considered as follows: cervical, thoracic, lumbar, sacral (fused into a sacrum) and coccyx. The spine performs a vital functions of backing, protection of spinal cord and is involved in the activities of torso and head.

Owing to the fact, spine functionalities appear to be the fundamental movement-wise operations, the science has always been keenly paying attention to the survey of more rapid, substantive and accurate solutions of spine analysis.
 
Whenever the speech is about the analysis of medical images we always refer to work with the images themselves in final extend, obviously meaning assay of certain set of numbers. Typically, the most challenging problems in analyses are classification of objects, as well as segmentation of different parts of human body in various image modalities.  
 
The advent and promotion of AI, as well as, growing of computing power, \cite{Pham2000} grants a huge impetus for medical research field itself. The raise of availability of different medical data sets had significantly increased including sundry vertebrae research papers. Researchers have applied AI to automatically recognizing complex patterns in imaging data and providing quantitative assessments of different characteristics.  
 
Jointly automatic vertebrae segmentation, precise analysis of the spinal structures is an essential condition in all the clinical applications of spinal imaging. Knowledge of the detailed shape of individual vertebrae can considerably aid early diagnosis, surgical planning and follow-up assessment of a number of spinal pathologies, such as degenerative disorders, spinal deformities, trauma and tumors. Classification and segmentation of fractured vertebrae by computer-assisted techniques may therefore provide additional support to diagnosis and treatment of vertebral fractures. 

In the thesis work I proposed two-stage approach for the automatic vertebrae segmentation. Whereas the first stage is so called detection and the second one localization. 
Detection part reveals the occurrence of vertebrae on the input CT images. Afterwards identification part recognises the concrete vertebrae within corresponding region-of-interest.