\chapter{Introduction}
\label{ch:introduction}


The main part of the human axial skeleton consists of 33-34 vertebrae which coherently connected to each other in an upright position \cite{Ahlberg2005}. Vertebrates types can be considered as follows: cervical, thoracic, lumbar, sacral (fused into a sacrum) and coccyx. The spine performs a vital functions of backing, protection of spinal cord and is involved in the activities of torso and head.

Owing to the fact, spine functionalities appear to be the fundamental movement-wise operations, the science has always been keenly paying attention to the survey of more rapid, substantive and accurate solutions of spine analyzes.
 
Whenever the speech is about the analysis of medical images the variety of which is more than huge we always by the end rely to work with the images themselves, obviously meaning certain set of numbers. Typically, the most challenging problems are classification of objects, as well as segmentation of different parts of human body in various images types.  
 
With the advent and promotion of AI, as well as increasing of computing power, \cite{Pham2000} medical research field itself as well as the availability of different medical data sets had significantly increased including various vertebrae research papers relied on SOTA deep learning models fited on open source data sets. Researchers have applied AI to automatically recognizing complex patterns in imaging data and providing quantitative assessments of different characteristics.  
 
In automatic vertebrae segmentation, precise analysis of the spinal structures from medical images is an essential tool in many clinical applications of spinal imaging. Knowledge of the detailed shape of individual vertebrae can considerably aid early diagnosis, surgical planning and follow-up assessment of a number of spinal pathologies, such as degenerative disorders, spinal deformities, trauma and tumors, as well as for the evaluation of vertebral fractures. Segmentation and classification of fractured vertebrae by computer-assisted techniques may therefore provide additional support to diagnosis and treatment of vertebral fractures. 

Jointly the work it is proposed two-stage approach for the automatic vertebrae segmentation. Whereas the first stage is so called detection and the second one localization. 
Detection part obviously detects the occurrence of vertebrae in the CT images approaching 3D samples. Thereby identification part identifies the concrete vertebrae within corresponding region-of-interest using 2D slices.