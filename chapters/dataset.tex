\chapter{Data set}
\label{ch:data_set}

\section{Data description}
%TODO change
The data set consists of spine-focused CT scans of 125 patients with varying types of pathologies. 

In the case of the detection model the dense labelling contains two values: 0 representing background and 1 representing vertebrae. 

The identification model’s dense labelling contains values from 0 to 26. Whilst 0 representing background, 1 representing C1 vertebrae up to 26 representing S2 vertebrae.

\begin{figure}[h]
    \centering \includegraphics[width=9cm]{images/labeled_data.png}
    \caption {(a) shows a dense labelling, (b) shows an example of a sample used to train the detection model, (c) shows an example of a sample used to train the identification model. Note: The size of the sample is 8 x 80 x 320, if the original scan is not larger enough to fill those dimensions some padding is added, as can be seen at the top.}
    \label{fig:labeled_data}
\end{figure}


\subsection{Data augmentation for detection model}
%TODO add samples
Unto passing particular batch of scans to detection model each image was represented as 5 cropped samples intruding that at least 4 of them contain some vertebrae pixels. In the meantime each sample scan was sized as $64x64x80$, meaning 64 pixels by 64 pixels by 80 channels. Withal accordingly each sample has an accompanying dense labelling, containing 0s (for background) and 1s (for vertebrae) of the same size.    

\subsection{Data augmentation for identification model}
%TODO change
%TODO add samples
To train the identification network it was applied cropping of samples sized $8x80x320$. In total generate 100 cropped samples for each scan in the training set enforcing that all of them capture some vertebrae pixels, each with a corresponding dense labelling of size $80x320$, representing the labelling for the 4th slice of the input sample.